\documentclass[12pt]{article}

\usepackage[letterpaper,margin=1.0in]{geometry}
\usepackage{url}

\pagestyle{empty}

% No paragraph indent or paragraph skip
\parindent=0pt \parskip=0pt

\begin{document}

\centerline{\bf Massively-parallel simulation of reduced-scaling ground and excited state coupled-cluster methods}

\vspace{12pt}

\centerline{ {Marjory C. Clement}$^{\rm a*}$ and {Edward F. Valeev}$^{\rm b}$}

\vspace{12pt}

\centerline{$^{\rm a}$Department of Chemistry, Virginia Tech}
\centerline{Blacksburg, VA United States}

\vspace{12pt}

\centerline{$^{\rm b}$Department of Chemistry, Virginia Tech}
\centerline{Blacksburg, VA United States}

\vspace{12pt}

\vspace{24pt}

{\bf Abstract:} To greatly increase the reach of conventional many-body techniques like
coupled-cluster (CC) we must employ numerical approximations that reduce the prohibitive high-order
computational complexity as well as deploy our reduced-scaling CC algorithms on the largest machines
available. With these complementary ideas in mind we implemented a simulation framework for reduced-scaling
CC methods based on the massively-parallel implementation
of ground- and excited-state CC methods in the re-engineered Massively Parallel Quantum Chemistry (MPQC)
package [1]. Although the current implementation retains the high complexity of the conventional CC methods,
this allows rapid exploration of reduced-scaling ansatz and provides a production-quality
solver component for reduced-scaling massively parallel coupled-cluster currently under development. 

Both the ground and excited state approaches can utilize arbitrary compressions of the cluster amplitudes,
including those based on pair-natural orbitals (PNOs) [2,3]. The idea behind PNOs
is quite simple: for each pair of electrons in a system,
only a few orbital products contribute substantially
to the pair-correlation wave function and pair energy, with the number of PNOs per pair essentially independent of the system size.
Furthermore, the PNOs determined using an approximate
first-order guess provide a useful subspace for solving the coupled-cluster equations.
Using the PNOs as the template, we devise similar compressions to the right- and left-hand vectors in EOM-CCSD.
The robust convergence of the simulated reduced-scaling CCSD(T) and EOM-CCSD energies will be demonstrated
with respect to the single rank-truncation parameter for systems of unprecedented size.

\vspace{12pt}

\vspace{12pt}

\parindent=0pt
{\bf References}

[1] ``MPQC4: Massively Parallel Quantum Chemistry'', Edward F. Valeev, Cannada 
A. Lewis, Chong Peng, Justus A. Calvin, Jinmei Zhang,
\url{https://github.com/valeevgroup/mpqc4}.

[2] C. Edmiston, M. Krauss, ``Configuration-Interaction Calculation of $H_3$ and $H_2$'',
J. Chem. Phys., {\bf 1965}, {\em 42}, 1119-1120

[3] Frank Neese, Frank Wennmohs, Andreas Hansen, ``Efficient and accurate local approximations
to coupled-electron pair approaches: An attempt to revive the pair natural orbital
method'', J. Chem. Phys., {\bf 2009}, {\em 130}, 114108

\end{document}
\documentclass[12pt]{article}

\usepackage[letterpaper,margin=1.0in]{geometry}
\usepackage{url}

\pagestyle{empty}

% No paragraph indent or paragraph skip
\parindent=0pt \parskip=0pt

\begin{document}

\centerline{\bf Simulation of massively-parallel PNO-based CCSD(T) and EOM-CCSD}

\vspace{12pt}

\centerline{ {Marjory C. Clement}$^{\rm a*}$ and {Edward F. Valeev}$^{\rm b}$}

\vspace{12pt}

\centerline{$^{\rm a}$Department of Chemistry, Virginia Tech}
\centerline{Blacksburg, VA United States}

\vspace{12pt}

\centerline{$^{\rm b}$Department of Chemistry, Virginia Tech}
\centerline{Blacksburg, VA United States}

\vspace{12pt}

\vspace{24pt}

{\bf Abstract:} 

% All quantum chemical methods development research is guided
% by a common goal: to develop efficient many-body methods for calculating and
% predicting molecular properties of interest. There are a number of different
% techniques and strategies employed in pursuit of computational efficiency,
% but most fall into one of two categories: reduced scaling and increased
% parallelization. The first category contains those techniques that seek to
% produce methods whose computational cost scales, ideally, linearly or
% sub-linearly with the size of the system, instead of super-linearly. The
% second category focuses on developing and implementing methods
% in a highly parallel manner. Although the two categories offer different
% approaches in pursuit of the same goal, they are not mutually exclusive;
% rather, they are complementary ideas that, together, are more effective
% than either one alone.


Nearly all quantum chemical methods development research seeks to increase
computational efficiency either by developing reduced-scaling methods
or by implementing current methods in a more highly parallelized manner.
A combination of these two tactics can, however, lead to more increases
in efficiency than either one alone. This current work explores one such
synthesis. We have simulated a pair natural orbital (PNO) based
coupled cluster result within the highly parallel framework of the
Massively Parallel Quantum Chemistry (MPQC) package [1]. The use of PNOs
allows us to decrease the scaling of highly accurate methods while
tightly controlling the error incurred The idea behind PNOs is simple:
for each pair of occupied molecular orbitals, certain unoccupied
molecular orbitals will contribute more substantially to the electronThese reduced-scaling methods
can then be readily implemented in a highly-parallel manner   using pair natural orbitals (PNOs) to decrease the scaling of
certain highly accurate methods while carefully controlling the loss in
accuracy that comes as a result

% This current work explores one example of the synthesis of reduced-scaling
% methods with increased parallelization
% such a synthesis, the implementation of a massively-parallel pair natural
% orbital-based coupled cluster code. The idea behind pair natural orbitals (PNOs)
% is quite simple: for each pair of molecular orbitals in a system,
% only some of the remaining molecular orbitals will contribute substantially
% to the calculation of the correlation energy. If we can determine which of
% the molecular orbitals are most important for each pair, then we can neglect
% the rest, leading, in many cases, to a respectable decrease in computational
% cost. In addition, PNOs have the added benefit of being system size-independent [1].
% That is, regardless of the size of the system, the number of PNOs associated
% with each pair will be esentially constant, so methods that employ PNOs
% enjoy reduced scaling relative to their counterparts applied to the
% entire orbital space. This present work further increases the efficiency
% that can be gained using PNOs by describing a
% massively-parallel implementation of a PNO-based coupled cluster method
% within the Massively Parallel Quantum Chemistry package [2].


\vspace{12pt}

\vspace{12pt}

\parindent=0pt
{\bf References}

[1] Fabijan Pavosevic, Peter Pinski, Christoph Riplinger, Frank Neese, Edward F.
Valeev, ``SparseMaps - A systematic infrastructure for reduced-scaling
electronic structure methods. IV. Linear-scaling second-order explicitly
correlated energy with pair natural orbitals'', J. Chem. Phys., {\bf 2016},
{\em 144}, 144109

[2] ``MPQC4: Massively Parallel Quantum Chemistry'', Edward F. Valeev, Cannada 
A. Lewis, Chong Peng, Justus A. Calvin, Jinmei Zhang,
\url{https://github.com/valeevgroup/mpqc4}.

\end{document}
